% Options for packages loaded elsewhere
\PassOptionsToPackage{unicode}{hyperref}
\PassOptionsToPackage{hyphens}{url}
%
\documentclass[
]{article}
\title{Advanced Esophageal Cancer}
\usepackage{etoolbox}
\makeatletter
\providecommand{\subtitle}[1]{% add subtitle to \maketitle
  \apptocmd{\@title}{\par {\large #1 \par}}{}{}
}
\makeatother
\subtitle{Jonathan Salo MD}
\author{\href{https://atriumhealth.org/medical-services/specialty-care/cancer-care/esophageal-cancer/meet-the-team\#}{Levine
Cancer Institute}}
\date{Click Right Arrow → to Begin}

\usepackage{amsmath,amssymb}
\usepackage{lmodern}
\usepackage{iftex}
\ifPDFTeX
  \usepackage[T1]{fontenc}
  \usepackage[utf8]{inputenc}
  \usepackage{textcomp} % provide euro and other symbols
\else % if luatex or xetex
  \usepackage{unicode-math}
  \defaultfontfeatures{Scale=MatchLowercase}
  \defaultfontfeatures[\rmfamily]{Ligatures=TeX,Scale=1}
\fi
% Use upquote if available, for straight quotes in verbatim environments
\IfFileExists{upquote.sty}{\usepackage{upquote}}{}
\IfFileExists{microtype.sty}{% use microtype if available
  \usepackage[]{microtype}
  \UseMicrotypeSet[protrusion]{basicmath} % disable protrusion for tt fonts
}{}
\makeatletter
\@ifundefined{KOMAClassName}{% if non-KOMA class
  \IfFileExists{parskip.sty}{%
    \usepackage{parskip}
  }{% else
    \setlength{\parindent}{0pt}
    \setlength{\parskip}{6pt plus 2pt minus 1pt}}
}{% if KOMA class
  \KOMAoptions{parskip=half}}
\makeatother
\usepackage{xcolor}
\IfFileExists{xurl.sty}{\usepackage{xurl}}{} % add URL line breaks if available
\IfFileExists{bookmark.sty}{\usepackage{bookmark}}{\usepackage{hyperref}}
\hypersetup{
  pdftitle={Advanced Esophageal Cancer},
  pdfauthor={Levine Cancer Institute},
  hidelinks,
  pdfcreator={LaTeX via pandoc}}
\urlstyle{same} % disable monospaced font for URLs
\usepackage[margin=1in]{geometry}
\usepackage{graphicx}
\makeatletter
\def\maxwidth{\ifdim\Gin@nat@width>\linewidth\linewidth\else\Gin@nat@width\fi}
\def\maxheight{\ifdim\Gin@nat@height>\textheight\textheight\else\Gin@nat@height\fi}
\makeatother
% Scale images if necessary, so that they will not overflow the page
% margins by default, and it is still possible to overwrite the defaults
% using explicit options in \includegraphics[width, height, ...]{}
\setkeys{Gin}{width=\maxwidth,height=\maxheight,keepaspectratio}
% Set default figure placement to htbp
\makeatletter
\def\fps@figure{htbp}
\makeatother
\setlength{\emergencystretch}{3em} % prevent overfull lines
\providecommand{\tightlist}{%
  \setlength{\itemsep}{0pt}\setlength{\parskip}{0pt}}
\setcounter{secnumdepth}{-\maxdimen} % remove section numbering
\usepackage{float}
\let\origfigure\figure
\let\endorigfigure\endfigure
\renewenvironment{figure}[1][2] {
    \expandafter\origfigure\expandafter[H]
} {
    \endorigfigure
}
\ifLuaTeX
  \usepackage{selnolig}  % disable illegal ligatures
\fi

\begin{document}
\maketitle

\hypertarget{introduction}{%
\section{Introduction}\label{introduction}}

I'm Dr Jonathan Salo, a GI Cancer Surgeon at the Levine Cancer Institute
in Charlotte, North Carolina.

If you're viewing this video, chances are that you or someone close to
you has encountered esophageal cancer and is contemplating treatment.

\begin{center}\rule{0.5\linewidth}{0.5pt}\end{center}

For a refresher, esophageal cancer, as it grows, can tends to make it
difficult for patients to swallow.

So patients with esophageal cancer fit into two main groups:

\begin{itemize}
\tightlist
\item
  A small group who don't have any difficulty eating that have
  \emph{early} stage disease
\item
  Majority of patients who have some difficulty eating or may have
  weight loss who have \emph{advanced} disease.
\end{itemize}

\begin{center}\rule{0.5\linewidth}{0.5pt}\end{center}

Among those with early stage disease, there are two categories:

\begin{itemize}
\tightlist
\item
  Superficial -\textgreater{} Treated without surgery
\item
  Localized -\textgreater{} Treated with surgery alone
\end{itemize}

For more information about early stage esophageal cancer, there is a
link above and in the description.
\href{06-Eso_Early.html}{Non-obstructing esophageal tumors}

\begin{center}\rule{0.5\linewidth}{0.5pt}\end{center}

This video will focus on \emph{advanced} esophageal cancer, which
consists of two categories: Locally Advanced and Metastatic.

\begin{itemize}
\tightlist
\item
  Locally Advanced -\textgreater{} T3M0
\item
  Metastatic -\textgreater{} M1
\end{itemize}

\begin{center}\rule{0.5\linewidth}{0.5pt}\end{center}

If this terminology is not familiar to you, please refer to our video on
Esophageal Cancer Diagnosis and Staging. There is a link above and in
the description below.

\href{02_Eso_Dx_Staging.html}{Esophageal Cancer Diagnosis and Staging}

\begin{center}\rule{0.5\linewidth}{0.5pt}\end{center}

Locally-advanced tumors are usually T3, meaning that the tumor has grow
through the wall of the esophagus \emph{AND} there is \emph{no} signs of
spread to other organs, so they are M0

\hypertarget{section}{%
\subsection{\texorpdfstring{\includegraphics[width=0.9\linewidth]{images/t3n01Artboard 3@4x}}{}}\label{section}}

Metastatic esophageal cancer is a case where there has been spread to
other organs such as the liver or lungs. These are considered stage M1.

\begin{center}\rule{0.5\linewidth}{0.5pt}\end{center}

To review: advanced esophageal cancer usually is accompanied by
difficulty eating, and consists of two caregories:

Locally Advanced cancers are T3 and M0 and usually Stage III

Metastatic cancers are M1 and considered Stage IV

\begin{center}\rule{0.5\linewidth}{0.5pt}\end{center}

\hypertarget{treatment}{%
\section{Treatment}\label{treatment}}

For patients with Locally Advanced tumors that are T3 and M0, the usual
treatment is a combination of chemotherapy, radiation, and surgery
called trimodality therapy. There is a link here for a video that
discussed the treatment of Locally Advanced tumors.

\begin{center}\rule{0.5\linewidth}{0.5pt}\end{center}

For patients with Metastatic tumors that are M1, the usual treatment is
chemotherapy. In most cases, this is given intravenously through a
central venous port. Chemotherapy is administered under the care of a
Medical Oncologist.

\begin{center}\rule{0.5\linewidth}{0.5pt}\end{center}

In either case, locally advanced or metastatic, chemotherapy is used as
a part of the initial treatment.

In addition, for many patients with advanced cancers, nutrition is a
challenge because the tumor can make it difficult to eat.

\begin{center}\rule{0.5\linewidth}{0.5pt}\end{center}

Once the staging studies are complete, you and your esophageal cancer
treatment team can design a treatment plan for you and your cancer.

At the end of this video there will be links to additional videos which
address locally-advanced esophageal cancer and metastatic esophageal
cancer. This way, if you already know the results of the staging
studies, you can view specific information.

\begin{center}\rule{0.5\linewidth}{0.5pt}\end{center}

I hope you have found this video helpful. This videos and others like it
are designed to

educate patients and families about esophageal cancer

and equip them for their discussions with their esophageal cancer care
team.

As always, these videos are no substitute for expert medical advice.

\begin{center}\rule{0.5\linewidth}{0.5pt}\end{center}

Feel free to leave a comment or a question, or if you have suggestions
for future videos.

If you or a family member have had an encounter with esophageal cancer,
I would love to hear about your experience, so please take a minute to
leave a comment below.

We're constantly creating new videos, so please subscribe to be notified
of new videos when we post them.

\begin{center}\rule{0.5\linewidth}{0.5pt}\end{center}

Here are some additional videos you may find helpful:

\href{14-Eso_Locally_Advanced.html}{Locally Advanced Esophageal Cancer}
\href{16-Eso_Metastatic.html}{Metastatic Esophageal Cancer}

\begin{center}\rule{0.5\linewidth}{0.5pt}\end{center}

\end{document}
