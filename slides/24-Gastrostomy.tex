% Options for packages loaded elsewhere
\PassOptionsToPackage{unicode}{hyperref}
\PassOptionsToPackage{hyphens}{url}
%
\documentclass[
]{article}
\usepackage{amsmath,amssymb}
\usepackage{lmodern}
\usepackage{ifxetex,ifluatex}
\ifnum 0\ifxetex 1\fi\ifluatex 1\fi=0 % if pdftex
  \usepackage[T1]{fontenc}
  \usepackage[utf8]{inputenc}
  \usepackage{textcomp} % provide euro and other symbols
\else % if luatex or xetex
  \usepackage{unicode-math}
  \defaultfontfeatures{Scale=MatchLowercase}
  \defaultfontfeatures[\rmfamily]{Ligatures=TeX,Scale=1}
\fi
% Use upquote if available, for straight quotes in verbatim environments
\IfFileExists{upquote.sty}{\usepackage{upquote}}{}
\IfFileExists{microtype.sty}{% use microtype if available
  \usepackage[]{microtype}
  \UseMicrotypeSet[protrusion]{basicmath} % disable protrusion for tt fonts
}{}
\makeatletter
\@ifundefined{KOMAClassName}{% if non-KOMA class
  \IfFileExists{parskip.sty}{%
    \usepackage{parskip}
  }{% else
    \setlength{\parindent}{0pt}
    \setlength{\parskip}{6pt plus 2pt minus 1pt}}
}{% if KOMA class
  \KOMAoptions{parskip=half}}
\makeatother
\usepackage{xcolor}
\IfFileExists{xurl.sty}{\usepackage{xurl}}{} % add URL line breaks if available
\IfFileExists{bookmark.sty}{\usepackage{bookmark}}{\usepackage{hyperref}}
\hypersetup{
  pdftitle={Esophagectomy for Esophageal Cancer},
  pdfauthor={Levine Cancer Institute},
  hidelinks,
  pdfcreator={LaTeX via pandoc}}
\urlstyle{same} % disable monospaced font for URLs
\usepackage[margin=1in]{geometry}
\usepackage{graphicx}
\makeatletter
\def\maxwidth{\ifdim\Gin@nat@width>\linewidth\linewidth\else\Gin@nat@width\fi}
\def\maxheight{\ifdim\Gin@nat@height>\textheight\textheight\else\Gin@nat@height\fi}
\makeatother
% Scale images if necessary, so that they will not overflow the page
% margins by default, and it is still possible to overwrite the defaults
% using explicit options in \includegraphics[width, height, ...]{}
\setkeys{Gin}{width=\maxwidth,height=\maxheight,keepaspectratio}
% Set default figure placement to htbp
\makeatletter
\def\fps@figure{htbp}
\makeatother
\setlength{\emergencystretch}{3em} % prevent overfull lines
\providecommand{\tightlist}{%
  \setlength{\itemsep}{0pt}\setlength{\parskip}{0pt}}
\setcounter{secnumdepth}{-\maxdimen} % remove section numbering
\usepackage{float}
\let\origfigure\figure
\let\endorigfigure\endfigure
\renewenvironment{figure}[1][2] {
    \expandafter\origfigure\expandafter[H]
} {
    \endorigfigure
}
\ifluatex
  \usepackage{selnolig}  % disable illegal ligatures
\fi

\title{Esophagectomy for Esophageal Cancer}
\usepackage{etoolbox}
\makeatletter
\providecommand{\subtitle}[1]{% add subtitle to \maketitle
  \apptocmd{\@title}{\par {\large #1 \par}}{}{}
}
\makeatother
\subtitle{Jonathan Salo MD}
\author{\href{https://atriumhealth.org/medical-services/specialty-care/cancer-care/esophageal-cancer/meet-the-team\#}{Levine
Cancer Institute}}
\date{Click Right Arrow → to Begin}

\begin{document}
\maketitle

{
\setcounter{tocdepth}{2}
\tableofcontents
}
\hypertarget{introduction}{%
\section{Introduction}\label{introduction}}

Esophageal cancer can cause difficulty with swallowing, which can limit
your body's ability to get the nutrition you need in order to keep your
body healthy during cancer treatment.

\begin{center}\rule{0.5\linewidth}{0.5pt}\end{center}

I'm Dr Jonathan Salo, I'm a GI Cancer Surgeon. In this video you'll
learn about

• Essential building blocks for good nutrition

• Protein supplements

• Feeding tubes

\begin{center}\rule{0.5\linewidth}{0.5pt}\end{center}

For most people with esophageal cancer who have difficulty eating,
things can get worse during cancer therapy.

Chemotherapy and radiation can lead to a temporary `sunburn' on the
inside of the esphagus called radiation esophagitis. his usually gets
better after the radiation ends

\begin{center}\rule{0.5\linewidth}{0.5pt}\end{center}

One of the questions you will want to address with your esophageal
cancer care team is whether or not you need a feeding tube to help with
your nutrition during cancer treatment.

Your dietitian and physicians will evaluate your situation and made a
recommendation, but I'd like to show you some options.

\begin{center}\rule{0.5\linewidth}{0.5pt}\end{center}

\#GI Tract Anatomy

Normally, food passes from the mouth into the esophagus, and then into
the stomach. The stomach serves as a reservoir for food, to allow you to
eat a big Thanksgiving. The stomach starts digestion, and then after the
meal slowly allows small portions of food to pass into the small
intestines, where most of the digestion occurs.

\begin{center}\rule{0.5\linewidth}{0.5pt}\end{center}

For patients with cancer of the esophagus or stomach, the most common
feeding tube is a jejunostomy tube

\begin{center}\rule{0.5\linewidth}{0.5pt}\end{center}

For patients with cancer of the esophagus, there are two different kinds
of feeding tube which can be used:

A gastrostomy tube is placed into the stomach

A jejunostomy tube is placed in the small intestines

Your dietitian and physician will help you decide which tube is best for
your situation

\begin{center}\rule{0.5\linewidth}{0.5pt}\end{center}

Gastrostomy tube

• Placed into stomach

Jejunostomy tube

• Placed into small intestine

\begin{center}\rule{0.5\linewidth}{0.5pt}\end{center}

Feeding Gastrostomy

A gastrostomy tube allows feeding with a syringe, which can be done
several times per day.

When it's not being used, the gastrostomy tube can be hidden underneath
clothing.

For patient who later need surgery on the esophagus, it will be
necessary to remove the

gastrostomy tube and place a jejunostomy tube, as the stomach frequently
used to create a new

esophagus

\begin{center}\rule{0.5\linewidth}{0.5pt}\end{center}

A gastrostomy tube can be placed either by endoscopy, which is called a
PEG tube

A gastrostomy tube can also be placed by laparoscopy, which is usually
preferred if

surgery on the esophagus is planned in the future.

Your surgeon will help you decide which kind of tube is best for you.
This is especially important if you will need esophageal surgery in the
future, as the stomach is frequently used to make a new esophagus

\begin{center}\rule{0.5\linewidth}{0.5pt}\end{center}

A gastrostomy tube is generally placed as an outpatient, which means you
can go home the same day.

In some cases, a central venous port for chemotherapy is placed at the
same time as a gastrostomy tube.

\begin{center}\rule{0.5\linewidth}{0.5pt}\end{center}

The other type of feeding tube is a jejunostomy.

A jejunostomy tube tube is placed into the small intestines. Because the
small intestine is used to receiving food in small quantities, a
jejunostomy tube requires the use of a pump to deliver feedings
gradually over a matter of hours.

In general, feedings are done at night in order to allow you to be
active during the day

\begin{center}\rule{0.5\linewidth}{0.5pt}\end{center}

A jejunostomy tube is used in cases where it's not possible to place a
gastrostomy tube, such as when there is a tumor in the stomach. A
jejunostomy tube is routinely used after esophageal surgery, so in
patients who need help with nutrition prior to surgery, it makes sense
to put in a jejunostomy tube before surgery. The same tube can then be
used for nutrition both before and after surgery.

\begin{center}\rule{0.5\linewidth}{0.5pt}\end{center}

We have a link here to a video that explains more

about the jejunostomy tube

\begin{center}\rule{0.5\linewidth}{0.5pt}\end{center}

I hope this video has been helpful.

Here are links to other videos you may find helpful.

Feel free to subscribe in order to be notified about new videos

\end{document}
